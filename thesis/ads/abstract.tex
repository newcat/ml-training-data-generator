%!TEX root = ../dokumentation.tex

\pagestyle{empty}

\iflang{de}{%
\renewcommand{\abstractname}{\langabstract} % Text für Überschrift

\begin{otherlanguage}{english} % auskommentieren, wenn Abstract auf Deutsch sein soll
\begin{abstract}

As machine learning is used more and more, it is essential for academic institutions to be able to teach machine learning techniques to their students. However, most machine learning algorithms require large amounts of data to be trained, which is often hard to find. Aditionally, similar datasets are often needed for lectures and exams. To solve this problem, an application is needed that is able to generate large datasets. These datasets are to be generated according to a configurable model.

In this student research project, a web application is developed that allows the generation of arbitrary datasets. These datasets follow a specific model that is modelled by the user through a graphical user interface. The graphical user interface uses the dataflow programming concept. As there existed no open-source dataflow programming library that fulfilled all requirements of this project, a new one was developed as part of this project. Key parts to the library and the application are presented in this work.

Randomness is an essential part of data generation. Therefore, algorithms are presented in this work, which can be used to generate random numbers that follow a specific random distribution.

To verify the functionality of the developed application, unit tests and usability tests were developed and conducted. The development process as well as the results of those are also part of this work.

\end{abstract}
\end{otherlanguage} % auskommentieren, wenn Abstract auf Deutsch sein soll
}
