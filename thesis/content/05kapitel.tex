%!TEX root = ../dokumentation.tex

\chapter{Implementierung}

\section{BaklavaJS}

Eine der Anforderungen war es, das Modell visuell bearbeiten zu können. Dafür war es notwendig, einen Graph-Editor zu entwickeln. Ein solcher Editor erlaubt es, Knoten hinzuzufügen bzw. zu entfernen und sie miteinander zu verbinden.

Ein Knoten ist dabei wie eine mathematische Funktion: Er führt einen Algorithmus auf die Eingangsdaten aus und gibt die erzeugten Ausgangsdaten aus.

Visuell wird ein Knoten in BaklavaJS folgendermaßen dargestellt:


\section{Random Sampling / Custom Random}