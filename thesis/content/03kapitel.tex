%!TEX root = ../dokumentation.tex

\chapter{Grundlagen des Machine Learning}

\begin{itemize}
    \item Warum braucht man Trainingsdaten?
    \item Wie sehen Trainingsdaten aus?
    \item Was für Möglichkeiten gibt es, Trainingsdaten zu bekommen?
\end{itemize}

\begin{quote}
    TODO: Definition nach Mitchell
    Machine Learning is the study of computer algorithms that improve automatically through experience
\end{quote}

Das Ziel von maschinellem Lernen ist es, einen Algorithmus zu entwickeln, der anhand von Eingabedaten Ausgabedaten erzeugt. TODO

Die Vorgehensweise beim Trainieren kann hauptsächlich in zwei Gruppen eingeteilt werden: \textit{Supervised Learning} und \textit{Unsupervised Learning}. Beim Supervised Learning benötigt man in der Trainingsphase Daten, die sowohl aus den Eingabedaten, als auch aus den erwarteten Ausgabedaten bestehen. Der Algorithmus wird während des Trainings so angepasst, dass die Eingabedaten die erwarteten Ausgabedaten erzeugen. Häufig wird Supervised Learning für \textit{Klassifikation} oder \textit{Regression} eingesetzt. Bei der Klassifikation wird den Eingabedaten eine von mehreren vordefinierten Klassen zugewiesen. Bei der Regression dagegen wird 
