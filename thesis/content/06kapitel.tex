%!TEX root = ../dokumentation.tex

\chapter{Implementierung der Knoten}

Mit BaklavaJS wurde das Grundgerüst der Applikation erstellt. Auf BaklavaJS aufbauend wird im \ac{MLTDG} die Funktionalität mittels der verschiedenen Knoten implementiert. In diesem Kapitel wird zum einen vorgestellt, welche Knoten implementiert wurden. Zum anderen wird darauf eingegangen, wie die grafische Oberfläche für einstellbare Wahrscheinlichkeitsverteilungen implementiert wurde.

\section{Knoten-Arten}
Der Kern der Applikation bilden die verschiedenen Knoten: Mit ihnen kann das Modell aufgebaut werden. Dabei ist es wichtig, dass die Knoten die verschiedenen Anwendungsfälle abdecken.

Um herauszufinden, welche Knoten-Arten benötigt werden, wurden mehrere Beispiel-Modelle durchgespielt und evaluiert, mit welchen Knoten diese sich am besten umsetzen lassen. Beispielhaft ist hier ein Modell dargestellt, welches Daten für einen fiktiven Immobilienmarkt generieren kann:
\begin{algorithm}[H]
    \caption{Beispielmodell Immobilienmarkt}
    \begin{algorithmic}[1]
        \State Fläche = ZufallGauss($\mu = 80$, $\sigma = 20$)
        \State a = Mathematik(Division, $\textrm{Fläche}$, $40$)
        \State b = ZufallDiskret($-1: 10\%, 0: 80\%, 1: 10\%$)
        \State c = Mathematik(Addition, $a$, $b$)
        \State Räume = Mathematik(Maximum, $c$, $1$)
        \State d = $\textrm{Fläche} \cdot 2500 + 250 \cdot \textrm{Räume}$
        \State Preis = ZufallProzentual($d$, $5\%$)
    \end{algorithmic}
\end{algorithm}

Die Knoten lassen sich in folgende Kategorien unterteilen:
\begin{itemize}
    \item \textbf{Werteknoten} können benutzt werden, um den gleichen Wert an verschiedene andere Knoten weiterzugeben. Es gibt sie für die Datentypen \textit{Boolean}, \textit{Number} und \textit{String}. Zusätzlich gibt es den Index-Knoten. Dieser gibt den aktuellen Index innerhalb des Berechnungsprozesses aus. Wird ein Wert nur für einen Knoten verwendet, kann dieser in der Regel direkt an der Eingangsschnittstelle des entsprechenden Knotens eingestellt werden; ein Werteknoten muss dafür nicht verwendet werden.
    \item \textbf{Zufallsknoten}
    \begin{itemize}
        \item \textbf{Gleichverteilung} mit einstellbarem Minimum und Maximum
        \item \textbf{Normalverteilung} mit einstellbarem Mittelwert $\mu$ und Standardabweichung\nobreakspace $\sigma$
        \item \textbf{Exponentialverteilung} mit einstellbarem $\lambda$
        \item \textbf{Anpassbare Verteilung}: Bei diesem Knoten kann die Wahrscheinlichkeitsdichtefunktion über einen grafischen Editor eingestellt werden. Die Wahrscheinlichkeitsdichtefunktion kann sowohl diskret als auch kontinuierlich sein.
        \item \textbf{Prozentuale Abweichung}: Dieser Knoten nimmt einen Wert und addiert einen zufälligen Wert zwischen $\pm \, \, inputValue \cdot \frac{percentage}{100}$
    \end{itemize}
    \item \textbf{Berechnungsknoten}
    \begin{itemize}
        \item \textbf{Mathematik}: Der Mathematikknoten wendet eine einstellbare mathematische Funktion, wie zum Beispiel Addition, Division oder trigonometrische Funktionen auf ein oder zwei Eingabedaten an
        \item \textbf{Funktion}: Der Funktionsknoten kann benutzt werden, um eine beliebige Funktion in JavaScript zu implementieren. Alle Eingangswerte werden im \texttt{this} Objekt zur Verfügung gestellt. Der Code muss ein Objekt zurückgeben, welches die Namen der Ausgangsschnittstellen als Keys und die dazugehörigen Werte enthält.
        \item \textbf{Boolean}: Dieser Knoten vergleicht zwei numerische Werte und gibt das Resultat des Vergleichs aus.
        \item \textbf{Stringlist}: Der Stringlist-Knoten erlaubt es, einen String aus einer vorgegebenen Liste von Strings auszuwählen. Welcher String ausgegeben wird, kann über die Index-Eingangsschnittstelle gesteuert werden (die Indizes beginnen bei 0)
    \end{itemize}
    \item \textbf{Bedingungsknoten}: Bedingungsknoten können genutzt werden, um sicherzustellen, dass Ausgabedaten bestimmte Bedingungen erfüllen. Wenn an der Eingangsschnittstelle der Wert \texttt{false} anliegt, wird der aktuelle Datenpunkt neu berechnet.
    \item \textbf{Ausgabeknoten}: Jeder Ausgabeknoten repräsentiert eine Spalte in den generierten Ausgabedaten. Der Name der Spalte kann im Textfeld des Ausgabeknotens gesetzt werden.
\end{itemize}