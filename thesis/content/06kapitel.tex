%!TEX root = ../dokumentation.tex

\chapter{Ergebnisanalyse}

\section{Unit Tests}

\begin{itemize}
    \item Mocha \& Chai
    \item Aufbau eines Tests
    \item Beispieltests
\end{itemize}

\section{Usability-Tests}

Beim Entwickeln eines Produkts besteht die Gefahr, dass die Entwickler die Komplexität und mögliche Schwachstellen im Bereich der User Experience übersehen \cite{Witte2018}. Aus diesem Grund werden Usability-Tests durchgeführt. Diese sollen sicherstellen, dass das Produkt von den Anwendern möglichst einfach und effizient genutzt werden kann.

Laut Moser ist der Ablauf eines Usability-Tests in sechs Phasen gegliedert \cite{Moser2012}:
\begin{enumerate}
    \item Ziel und Zweck festlegen
    \item Untersuchungsdesign entwerfen
    \item Teilnehmer rekrutieren
    \item Evaluation vorbereiten
    \item Evaluation durchführen
    \item Resultate auswerten
\end{enumerate}

