%!TEX root = ../dokumentation.tex

\chapter{Einleitung}

Insg. 6 Seiten

In letzter Zeit werden lernende Algorithmen in immer mehr Bereichen eingesetzt \cite{mengesonnentag:2018}. Diese Algorithmen müssen mit Daten trainiert werden. Allerdings ist es schwierig, geeignete Datensätze zu finden. Dies gilt besonders für akademische Zwecke, da hierbei oft mehrere Datensätze ähnlicher Art benötigt werden, die aber leichte Unterschiede aufweisen. So kann in der Vorlesung mit einem Datensatz geübt werden und in der Klausuraufgabe ein ähnlicher, den Studenten aber bisher unbekannter Datensatz, verwendet werden.

Im Moment werden die Daten durch Formeln in einer Tabellenkalkulation erstellt. Mit dieser Vorgehensweise ist es allerdings kompliziert, Zusammenhänge darzustellen. Auch sind Randbedingungen, wie z. B. "mind. 10 positive Beispiele generieren" nur mit hohem Aufwand umsetzbar.

\section{Aufgabenstellung}

\section{Anforderungsanalyse}
