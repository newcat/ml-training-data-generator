%!TEX root = ../dokumentation.tex

\chapter{Einleitung}

Immer mehr Unternehmen wie zum Beispiel Google, Amazon und Facebook investieren verstärkt in Forschung im Bereich Maschinelles Lernen. Dadurch gewinnen lernende Algorithmen eine immer größer werdende Bedeutung \cite{mengesonnentag:2018}. Diese Algorithmen müssen mit Daten trainiert werden. Allerdings ist es schwierig, geeignete Datensätze zu finden. Dies gilt besonders für akademische Zwecke, da hierbei oft mehrere Datensätze ähnlicher Art benötigt werden, die aber leichte Unterschiede aufweisen sollen. So können Daten nach einer bestimmten Logik generiert werden, um dann testweise lernende Algorithmen auf die Trainingsdaten anzuwenden. Das trainierte Modell kann nun mit dem Modell verglichen werden, mit dem die Daten ursprünglich generiert wurden. Im Idealfall sind sich beide Modelle sehr ähnlich.

Bisher werden Daten durch Formeln in einer Tabellenkalkulation erzeugt. Mit dieser Vorgehensweise ist es allerdings sehr aufwändig, komplexe Zusammenhänge darzustellen. Diese Arbeit befasst sich mit der Entwicklung einer Applikation, die die eben beschriebene Vorgehensweise vereinfacht und verbessert.

\section{Aufgabenstellung}

Das Ziel ist es, eine Applikation zu entwickeln, die mittels einer grafischen Benutzeroberfläche bedient werden kann, um Daten vollsynthetisch, einfach, schnell und nach vorgegebenen Modellen zu generieren. Die Datengenerierung muss die Komplexität der bisher angewandten Methode verringern und mit neuen Funktionalitäten erweitert werden. Dadurch, dass die generierten Daten vor allem im Bereich des maschinellen Lernens eingesetzt werden, muss mit enorm großen Datenmengen, bis hin zu mehreren Millionen  Datensätzen, gerechnet werden.

\section{Anforderungsanalyse}
\label{sec:anforderungsanalyse}

Die Anforderungsanalyse hat einige Anforderungen ergeben, die in der zu entwickelnden Applikation umgesetzt werden sollen.

Die erste Anforderung ist die Visualisierung des Modells mit Editierfunktionen und umfasst die Grundfunktionalität der Applikation. Die Applikation soll es ermöglichen, mithilfe einer grafischen Benutzeroberfläche ein Modell zu erstellen, auf dessen Basis Trainingsdaten generiert werden können.

Bei bestimmten Aufgaben, wie zum Beispiel binärer Klassifikation, werden Trainingsdaten benutzt, die sowohl Positiv- als auch Negativ-Beispiele beinhalten. Unter Umständen kann es sein, dass die Generierung eines Positiv-Beispiels sehr unwahrscheinlich ist; gleichzeitig sollen die generierten Datensätze aber eine Minimalanzahl von positiven Beispielen beinhalten. Für diesen Zweck soll es möglich sein, eine minimale oder maximale Menge von Positiv- beziehungsweise Negativ-Beispielen anzugeben.

Da lernende Algorithmen teilweise enorm große Datenmengen benötigen, um ein Modell hinreichend trainieren zu können, muss ein großer Wert auf die Leistung der Datengenerierung gelegt werden. Es soll möglich sein, Daten mit mehreren Millionen Datensätzen zu generieren.

Aufgrund der besonderen Schwierigkeit, logische Zusammenhänge grafisch modellieren zu können, sind auch Usability-Tests von großer Bedeutung. Die Software soll mit einer kurzen Anleitung auch für Nutzer bedienbar sein, die sich in dem Kontext nicht gut auskennen.

Eine weitere Anforderung ist die Möglichkeit, generierte Traningsdaten zu visualisieren. Dies ist vor allem deshalb von großer Bedeutung, weil durch eine Visualisierung das erstellte Modell validiert werden kann. Aus den Rohdaten ist es schwer auf modellierte Zusammenhänge, Korrelationen oder andere Abhängigkeiten zu schließen. Hingegen kann eine Darstellung, wie zum Beispiel ein Streudiagramm, Aufschluss über die Abhängigkeitsstruktur zweier Variablen geben.

Da es möglich sein soll eine große Menge an Trainingsdaten zu generieren, stellt der Datenexport eine weitere entscheidende Anforderung dar. Die Trainingsdaten sollen in gängigen Formaten, wie zum Beispiel CSV oder JSON, exportiert werden, damit auch andere Applikationen problemlos die exportierten Daten importieren können.

Eine andere Anforderung bezüglich des Datenexports ist die Möglichkeit die generierten Daten in einem Format zu exportieren, das dedizierte \ac{ML}-Tools, wie beispielweise WEKA, unterstützen. Da es sich hierbei um eine Anforderung handelt, welche nicht die Grundfunktionaliät umfasst, wird sie mit vergleichsweise geringer Priorität eingestuft.

Im Folgenden werden die genannten Anforderungen noch einmal zusammengefasst:
\begin{itemize}
    \item Visualisierung des Modells mit Editierfunktion
    \item Generierung von Trainingsdaten mit Randbedingungen
    \item Effiziente Generierung von Trainingsdaten
    \item Durchführung von Usability-Tests
    \item Visualisierung der generierten Trainingsdaten
    \item Export der Trainingsdaten allgemeine Formate wie CSV oder JSON 
    \item Export für dedizierte \ac{ML}-Tools wie beispielsweise WEKA
\end{itemize}

\section{Struktur der Arbeit}

\todo{Überblick über die Kapitel}