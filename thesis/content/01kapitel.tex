%!TEX root = ../dokumentation.tex

\chapter{Einleitung}

In letzter Zeit werden lernende Algorithmen in immer mehr Bereichen eingesetzt \cite{mengesonnentag:2018}. Diese Algorithmen müssen mit Daten trainiert werden. Allerdings ist es schwierig, geeignete Datensätze zu finden. Dies gilt besonders für akademische Zwecke, da hierbei oft mehrere Datensätze ähnlicher Art benötigt werden, die aber leichte Unterschiede aufweisen. So kann in der Vorlesung mit einem Datensatz geübt werden und in der Klausuraufgabe ein ähnlicher, den Studenten aber bisher unbekannter Datensatz, verwendet werden.

Im Moment werden die Daten durch Formeln in einer Tabellenkalkulation erstellt. Mit dieser Vorgehensweise ist es allerdings kompliziert, Zusammenhänge darzustellen. Auch sind Randbedingungen, wie z. B. "mind. 10 positive Beispiele generieren" nur mit hohem Aufwand umsetzbar.

\section{Aufgabenstellung}

Das Ziel ist es eine Applikation zu entwickeln, die mittels einer grafischen Benutzeroberfläche bedient werden kann, um Daten vollsynthetisch, einfach, schnell und nach vorgegebenen Modellen zu generieren. Die Datengenerierung muss die Komplexität der bisher alternativen Methode verringern. Dadurch, dass die generierten Daten vor allem im Bereich Machine Learning eingesetzt werden, muss mit enorm großen Datenmengen, bis hin zu mehreren Millionen Datenpunkten pro Datensatz, gerechnet werden. Aus diesem Grund darf die Performance der Applikation nicht vernachlässigt werden.

\section{Anforderungsanalyse}
\label{sec:anforderungsanalyse}

\begin{itemize}
    \item Visualisierung des Modells mit Editierfunktion (ähnlich Netica)
    \item Generierung von Trainingsdaten mit Randbedingungen (z. B. \enquote{es müssen mind. 10 positive Beispiele generiert werden})
    \item Visualisierung der generierten Trainingsdaten
    \item Export der Trainingsdaten
    \item Export für dedizierte ML-Tools wie beispielsweise WEKA
    \item Export in allgemeine Formate wie CSV oder JSON 
\end{itemize}

\section{Struktur der Arbeit}

\todo{Überblick über die Kapitel}