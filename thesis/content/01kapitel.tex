%!TEX root = ../dokumentation.tex

\chapter{Einleitung}

In letzter Zeit werden lernende Algorithmen in immer mehr Bereichen eingesetzt \cite{mengesonnentag:2018}. Diese Algorithmen müssen mit Daten trainiert werden. Allerdings ist es schwierig, geeignete Datensätze zu finden. Dies gilt besonders für akademische Zwecke, da hierbei oft mehrere Datensätze ähnlicher Art benötigt werden, die aber leichte Unterschiede aufweisen. So kann in der Vorlesung mit einem Datensatz geübt werden und in der Klausuraufgabe ein ähnlicher, den Studenten aber bisher unbekannter Datensatz, verwendet werden.

Im Moment werden die Daten durch Formeln in einer Tabellenkalkulation erstellt. Mit dieser Vorgehensweise ist es allerdings kompliziert, Zusammenhänge darzustellen. Auch sind Randbedingungen, wie z. B. "mind. 10 positive Beispiele generieren" nur mit hohem Aufwand umsetzbar.

\section{Aufgabenstellung}

Das Ziel ist es eine Applikation zu entwickeln, die mittels einer grafischen Benutzeroberfläche bedient werden kann, um Daten vollsynthetisch, einfach, schnell und nach vorgegebenen Modellen zu generieren. Die Datengenerierung muss die Komplexität der bisher alternativen Methode verringern. Dadurch, dass die generierten Daten vor allem im Bereich Machine Learning eingesetzt werden, muss mit enorm großen Datenmengen, bis hin zu mehreren Millionen Datenpunkten pro Datensatz, gerechnet werden. Aus diesem Grund darf die Performance der Applikation nicht vernachlässigt werden.

\section{Anforderungsanalyse}
\label{sec:anforderungsanalyse}
Die Anforderungsanalyse hat einige Anforderungen ergeben, die mit der entwickelten Applikation umgesetzt werden sollen.

Die erste Anforderung ist die Visualisierung des Modells mit Editierfunktionen und umfasst die Grundfunktionalität der Applikation. Die Applikation soll es ermöglichen mithilfe einer grafischen Benutzeroberfläche ein Modell zu erstellen, auf dessen Basis Trainingsdaten generiert werden können.

Es soll möglich sein die Menge an Positiv-Beispielen mit Randbedingungen einzuschränken.

Eine weitere Anforderung ist die Möglichkeit generierte Traningsdaten zu visualisieren. Dies ist vor allem deshalb von großer Bedeutung, weil durch eine Visualisierung das erstellte Modell validiert werden kann. Aus den Rohdaten ist es schwer auf modellierte Zusammenhänge, Korrelationen oder andere Abhängigkeiten zu schließen. Hingegen kann eine Darstellung wie z. B. ein Streudiagramm Aufschluss über die Abhängigkeitsstruktur zweier Variablen geben.

Da es möglich sein soll eine große Menge an Trainingsdaten zu generieren, stellt der Datenexport eine weitere entscheidende Anforderung dar. Die Trainingsdaten sollten in gängigen Formaten wie z. B. CSV oder JSON exportiert werden können, damit auch andere Applikationen problemlos die exportierten Daten importieren können.

Eine andere Anforderung bezüglich des Datenexports ist die Möglichkeit die generierten Daten in einem Format zu exportieren, das dedizierte ML-Tools wie beispielweise WEKA unterstützen. Da es sich hierbei um eine Anfoderung handelt, welche nicht die Grundfunktionaliät umfasst, wird sie mit vergleichsweise geringer Priorität eingestuft. 

