%!TEX root = ../dokumentation.tex

\chapter{Stand der Technik}

Themenumfeld und Stand der Technik erörtern (insg. 3 Seiten)

\section{Datengeneratoren}

Es gibt einige Datengeneratoren; diese sind jedoch extrem auf einen bestimmten Anwendungsfall zugeschnitten.

\begin{itemize}
    \item \textbf{Scikit-Learn}: Die Machine Learning Bibliothek scikit-learn bietet Funktionalität, um Beispieldaten zu generieren. Allerdings bietet sie nur vorgefertigte Funktionen, die bestimmte Muster zufällig generieren. Die Funktionen lassen sich kaum anpassen und es ist nicht möglich, damit ein komplexes Modell abzubilden. \cite{scikit-learn:paper, scikit-learn:generator}
    \item \textbf{SynthOSNdataGenerator}: Dieses Programm ist in der Lage, die Struktur eines sozialen Netzwerks zu erzeugen. Auch hier ist die Funktionalität sehr auf der Anwendungsfall angepasst und nicht universell genug, um als Basis für die in dieser Arbeit zu entwickelnde Applikation zu dienen. \cite{synthosndatagenerator}
    \item \textbf{log-synth}: Das Tool log-synth ist in der Lage, künstliche Log-Dateien zu erzeugen. Es erlaubt eine sehr flexible Konfiguration und hat bereits viele eingebaute Funktionen, jedoch ist es sehr schwierig, Zusammenhänge zwischen mehreren Eigenschaften zu modellieren. Dadurch kann auch dieses Tool nicht als Grundlage für die Arbeit genommen werden. \cite{logsynth}
\end{itemize}

\section{Node Editor}

Eine der Anforderungen ist eine graphische Oberfläche zur Modellierung des Modells zur Datengenerierung. Dafür wurde entschieden, das \textit{\ac{DFP}} Konzept umzusetzen.

Beim \ac{DFP} wird das Programm in einem gerichteten Graphen dargestellt. Jeder Knoten bildet eine Funktion ab und hat Ein- und Ausgabeschnittstellen. Über die Eingabeschnittstellen erhält die Funktion des Knotens die benötigten Parameter; das Resultat der Funktion wird über die Ausgabeschnittstellen anderen Knoten zur Verfügung gestellt. Die Schnittstellen von Knoten können über Kanten verbunden werden. Kanten symbolisieren einen Datenfluss von einer Aus- zu einer Eingabeschnittstelle \cite{dataflow}. Einer der Vorteile von \ac{DFP} ist, dass Abhängigkeiten von Operationen durch die grafische Programmierung direkt ersichtlich sind. Auch sind Programme, die mit \ac{DFP} erstellt wurden, parallelisierbar, solange die Knoten bei der Berechnung keine Seiteneffekte haben \cite{dataflow}.

Der Node-Editor ist die grafische Oberfläche, die es erlaubt, das \ac{DFP}-Programm zu bearbeiten. Technisch gesehen ist ein Node-Editor ein Editor für gerichtete Graphen. Es hat sich aber bei Anwendersoftware der Begriff \textit{Node-Editor} durchgesetzt \cite{nodeeditor:blender, nodeeditor:maya}. Deshalb werden in dieser Arbeit die Begriffe \textit{Node-Editor} und \textit{Graph-Editor} synonym verwendet.

\todo{Überarbeiten}
Für die Entwicklung des Node Editors wird eine Möglichkeit benötigt spezifische grafische Elemente anzeigen zu können. Herkömmliche HTML-Elemente sind unvorteilhaft dafür zu benutzen, weil es ihnen an Flexibilität und Interagierbarkeit fehlt. Da der Benutzer in der Lage sein soll einzelne Nodes flexibel im Raum verschieben, konfigurieren und verbinden zu können, wird eine Zeichenfläche benötigt z.B. durch ein Canvas- oder SVG-Element.

Canvas ist ein HTML-Element, das verwendet wird, um Grafiken auf einer Webseite zu zeichnen. Es handelt sich um eine Bitmap auf der mittels einer grafischen Anwendungsprogrammierschnittstelle (API) pixelweise gezeichnet werden kann.

SVG-Elemente werden auch dazu verwendet Grafiken auf einer Zeichenfläche anzuzeigen, jedoch werden Objekte mit Vektoren dargestellt. Das HTML <svg>-Element ist ein Container, der die Nutzung von SVG-Grafiken auf Webseiten unterstützt.

\subsection{Node-Editoren für JavaScript}

\todo{KO-Kriterien für Node-Editor}

\begin{itemize}
    \item Steuerelemente in den Knoten
    \item Steuerelemente in den Eingangsschnittstellen
    \item Sidebar für komplexe Einstellungen von Knoten
    \item Ausführliche Dokumentation
\end{itemize}

Es gibt bereits einige offene Bibliotheken für JavaScript, die die Funktionalität eines Node-Editors bieten. Im Folgenden werden einige dieser Bibliotheken vorgestellt.

\begin{itemize}
    \item \textbf{litegraph.js} ist eine Canvas-basierte Bibliothek. Damit bietet sie eine hohe Leistung. Es ist allerdings aufwändig, eigene Steuerelemente einzufügen, da diese selber gezeichnet werden müssen. Auch ist die Dokumentation spärlich. \cite{litegraph}
    \item \textbf{ThreeNodes.js} ist auf die Bearbeitung von WebGL-Szenen ausgelegt. Die Bibliothek basiert auf HTML und SVG und bietet laut eigener Aussage eine einfache Schnittstelle, um neue Knoten hinzuzufügen. Leider gibt es keine Dokumentation, außerdem wurde das Projekt zuletzt im Mai 2016 aktualisiert. \cite{threenodes}
    \item \textbf{vue-blocks} ist eine rudimentäre Umsetzung eines Node-Editors mit VueJS. Allerdings sind die Möglichkeiten der Bibliothek sehr beschränkt; es gibt beispielsweise keine Möglichkeit, eigene Steuerelemente in den Nodes anzuzeigen. \cite{vueblocks}
    \item \textbf{Nodes} ist eine mit VueJS entwickelte Bibliothek, die auf dem HTML5-Canvas aufbaut. Der Autor schreibt allerdings selber, dass die Bibliothek mehr als Experiment gedacht war und man lieber eine eigene Bibliothek entwickeln solle anstatt diese Bibliothek zu verwenden. \cite{nodes}
    \item \textbf{Linker} baut auf HTML und SVG auf. Leider ist is mit Linker nicht möglich, Steuerelemente in die Nodes einzufügen. \cite{linker}
\end{itemize}

Da keine der Bibliotheken alle der oben genannten Kriterien erfüllt, wurde entschieden, eine eigene Bibliothek zu schreiben. Diese vereint die Vorteile der vorgestellten Bibliotheken und wird in Kapitel \ref{sec:grapheditor} (Seite \pageref{sec:grapheditor}) näher vorgestellt.