%!TEX root = ../dokumentation.tex

\chapter{Stand der Technik}

Themenumfeld und Stand der Technik erörtern (insg. 3 Seiten)

\section{Datengeneratoren}

Es gibt einige Datengeneratoren; diese sind jedoch extrem auf einen bestimmten Anwendungsfall zugeschnitten.

\begin{itemize}
    \item \textbf{Scikit-Learn}: Die Machine Learning Bibliothek scikit-learn bietet Funktionalität, um Beispieldaten zu generieren. Allerdings bietet sie nur vorgefertigte Funktionen, die bestimmte Muster zufällig generieren. Die Funktionen lassen sich kaum anpassen und es ist nicht möglich, damit ein komplexes Modell abzubilden. \cite{scikit-learn:paper, scikit-learn:generator}
    \item \textbf{SynthOSNdataGenerator}: Dieses Programm ist in der Lage, die Struktur eines sozialen Netzwerks zu erzeugen. Auch hier ist die Funktionalität sehr auf der Anwendungsfall angepasst und nicht universell genug, um als Basis für die in dieser Arbeit zu entwickelnde Applikation zu dienen. \cite{synthosndatagenerator}
    \item \textbf{log-synth}: Das Tool log-synth ist in der Lage, künstliche Log-Dateien zu erzeugen. Es erlaubt eine sehr flexible Konfiguration und hat bereits viele eingebaute Funktionen, jedoch ist es sehr schwierig, Zusammenhänge zwischen mehreren Eigenschaften zu modellieren. Dadurch kann auch dieses Tool nicht als Grundlage für die Arbeit genommen werden. \cite{logsynth}
\end{itemize}

\section{Node Editor}

Für die Entwicklung des Node Editors wird eine Möglichkeit benötigt spezifische grafische Elemente anzeigen zu können. Herkömmliche HTML-Elemente sind unvorteilhaft dafür zu benutzen, weil es ihnen an Flexibilität und Interagierbarkeit fehlt. Da der Benutzer in der Lage sein soll einzelne Nodes flexibel im Raum verschieben, konfigurieren und verbinden zu können, wird eine Zeichenfläche benötigt z.B. durch ein Canvas- oder SVG-Element.

Canvas ist ein HTML-Element, das verwendet wird, um Grafiken auf einer Webseite zu zeichnen. Es handelt sich um eine Bitmap auf der mittels einer grafischen Anwendungsprogrammierschnittstelle (API) pixelweise gezeichnet werden kann.

SVG-Elemente werden auch dazu verwendet Grafiken auf einer Zeichenfläche anzuzeigen, jedoch werden Objekte mit Vektoren dargestellt. Das HTML <svg>-Element ist ein Container, der die Nutzung von SVG-Grafiken auf Webseiten unterstützt.

% TODO
Vorteile Nachteile HTML Canvas / SVG
Warum Entscheidung für SVG?

\subsection{Node-Editoren für JavaScript}

Es gibt bereits einige offene Bibliotheken für JavaScript, die die Funktionalität eines Node-Editors bieten. Im Folgenden werden einige dieser Bibliotheken inklusive deren Vor- und Nachteile vorgestellt.

\subsubsection*{litegraph.js}
\textit{litegraph.js} ist eine Canvas-basierte Bibliothek. Damit bietet sie eine hohe Leistung. Es ist allerdings aufwändig, eigene Steuerelemente einzufügen, da diese selber gezeichnet werden müssen. Auch ist die Dokumentation spärlich.

\subsubsection*{ThreeNodes.js}
Die Bibliothek \textit{ThreeNodes.js} ist auf die Bearbeitung von WebGL-Szenen ausgelegt.

\subsubsection*{vue-blocks}
\textit{vue-blocks} ist eine rudimentäre Umsetzung eines Node-Editors mit VueJS. Allerdings sind die Möglichkeiten der Bibliothek sehr beschränkt; es gibt beispielsweise keine Möglichkeit, eigene Steuerelemente in den Nodes anzuzeigen.

\subsubsection*{Nodes}
\textit{Nodes} ist eine mit VueJS entwickelte Bibliothek, die auf dem HTML5-Canvas aufbaut. Der Autor schreibt allerdings selber, dass die Bibliothek mehr als Experiment gedacht war und man lieber eine eigene Bibliothek entwickeln solle anstatt diese Bibliothek zu verwenden.

\subsubsection*{linker}
\textit{Linker} 