%!TEX root = ../dokumentation.tex

\chapter{Technologie Evaluation}

In diesem Kapitel werden die Entscheidungen im Hinblick auf die verwendeten Technologien vorgestellt.

\section{Plattform}

Da die verwendete Plattform Basis für alle weiteren zu treffenden Entscheidungen ist, muss diese zuerst gewählt werden. Für die zu entwickelnde Applikation gab es dabei folgende technische Anforderungen an die Plattform:

\begin{itemize}
    \item Grafische Benutzeroberfläche
    \item Effizientes Berechnen des Datensatzes aus dem Modell
    \item Möglichkeit zur Visualisierung von Daten
\end{itemize}

Die Applikation soll auf jeden Fall auf Windows laufen können, andere Betriebssysteme sollen nach Möglichkeit aber auch unterstützt werden. Aus diesen Anforderungen, sowie der Erfahrung der Teammitglieder, ergaben sich folgende Möglichkeiten:
\begin{itemize}
    \item C\#
    \item Java
    \item Web (HTML, CSS, JavaScript/TypeScript, Web Assembly)
\end{itemize}

\todo{Nur optimiert für Chrome}

\section{JavaScript / TypeScript}

\section{Chrome vs Electron}

Electron ist eine Bibliothek für die Entwicklung von Cross-Platform-Applikationen unter der Nutzung von Webtechnologien (\url{https://electronjs.org/docs/tutorial/about}). Mit Electron können Web-Applikationen entwickelt werden, die als Desktop-Applikation auf Windows, Linux oder MacOS laufen und Zugriff auf Betriebssystemfunktionalitäten wie zum Beispiel das Dateisystem haben.

\todo{Warum haben wir nicht Electron genommen?}