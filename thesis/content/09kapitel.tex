%!TEX root = ../dokumentation.tex

\chapter{Fazit}

Diese Arbeit handelt von der Entwicklung einer Applikation, um mittels einer grafischen Benutzeroberfläche Daten vollsynthetisch, einfach, schnell und nach vorgegebenen Modellen zu generieren. Um den Anforderungen gerecht zu werden, wurde für den \ac{MLTDG} parallel der Graph-Editor BaklavaJS entwickelt. Mit BaklavaJS ist es möglich benutzerdefinierte Knoten zu erstellen, um mit ihnen einen editierbaren Graphen aufzubauen. So konnten Node-Arten implementiert werden, die für die Generierung von Trainingsdaten von großer Bedeutung sind. Dazu zählen die Werte-, Zufalls-, Berechnungs, Bedingungs und Ausgabeknoten. Da die Zufallskomponente ein Grundbestandteil bei der Datengenerierung darstellt, wurden verschiedene Möglichkeiten entwickelt, um Zufallszahlen aus verschiedenen Wahrscheinlichkeitsverteilungen zu erzeugen, wie beispielsweise Gleich-, Exponential-, Normal- und benutzerdefinierte Verteilungen. Usability-Tests haben dazu beigetragen, hohe Qualität bei der Applikation sicherzustellen. Nutzern wurde dafür testweise die Aufgabe gegeben, nur mit Hilfe einer Anleitung, ein vordefiniertes Modell mit der Applikation zu erstellen. Unklarheiten konnten so erkannt werden und die Verbesserungen zu einer intuitiven Bedienung beitragen. Um die effiziente Generierung von großen Datenmengen möglich zu machen, wurden Web Worker implementiert. Mit Web Workern wurde es möglich Daten parallel in separaten Hintergrundthreads, ohne Beeinträchtigung der Anwendung, zu generieren. Die Anforderung, generierte Daten in einem Datenformat zu exportieren, das von dedizierten \ac{ML}-Tools, wie beispielsweise WEKA, unterstützt wird, konnte mit dem Datenexport zu \ac{CSV} abgedeckt werden, da nahezu alle \ac{ML}-Tools das Datenformat unterstützen. Lediglich die Anforderung eine minimale oder maximale Menge von Positiv- beziehungsweise Negativ-Beispielen angeben zu können, war aufgrund der parallelen Generierung in Web Workern nicht implementierbar. Über den ursprünglichen Kontext hinaus, Trainingsdaten für maschinelles Lernen zu generieren, kann der \ac{MLTDG} für die Generierung von allen Daten eingesetzt werden, die einer bestimmten Logik folgen. Beispielsweiserff könnte der \ac{MLTDG} dafür eingesetzt werden, Datensätze für Performance-Tests bei Datenbanken zu generieren.